\chapter{Introdução}

\section{Motivação}

A ascensão da Internet das Coisas (IoT) introduziu aos mais diferentes tipos de objetos
físicos o acesso à internet. Esses objetos, às vezes imperceptíveis no nosso dia a dia,
gerando agora um grande fluxo coletivo de dados em tempo real, que em conjunto às 
tecnologias de Aprendizagem de Maquina (ML) \cite{8001570}, tornou possível aplicações
cada vez mais independentes e inteligentes em diversas áreas como agricultura,
transporte, saúde, entre outras \cite{8883163}\cite{8358540}\cite{7474197}.

Além de suas características físicas, os objetos possuem sistemas eletrônicos embutidos, 
comumente conhecidos como sistemas embarcados. Esses sistemas são baseados em 
microcontroladores com sensores e/ou atuadores. Eles possuem poder computacional, 
conetividade, e energia limitados se
comparados aos dispositivos mais comuns, como smartfones e computadores. Devido a tais 
limitações, a indústria IoT impulsiona pesquisas que viabilizem a construção 
de sistemas inteligentes nos mais restritos cenários \cite{9221219}.

Nos últimos anos, tecnologias como LPWAN (Low Power Wide Area Network) tem tornado
realidade sistemas IoT com objetos 
distribuídos num grande espaço geográfico por oferecerem baixo consumo energético e 
conectividade a longas distâncias \cite{7815384}. Dentre essas tecnologias como
Sigfox e NB-IoT, LoRa (Longe Range) tem ganhado destaque entre elas por sua conectividade 
a distâncias quilométricas, bidirecionalidade, baixo consumo, alta capacidade de nós na 
rede, e por operar na faixa de frequência ISM (Industrial Scientific and Medical),
de livre licença para uso em diversos países \cite{9230597}.

Porém, de maneira simples, LoRa pode ser considerada tanto uma modulação de rádio 
frequência como uma camada física de transmissão de dados, algo que não garante ao 
desenvolvedor IoT um suporte de rede entre os nós e outras aplicações \cite{8474715}.
Com isso, na indústria se propõem diferentes protocolos a serem implementados por cima
da camada física da LoRa que estabelecem uma topologia de rede para o sistema.
Dentre os protocolos, o mais difundido atualmente é o LoRaWAN, um protocolo aberto 
proposto pela LoRa Alliance \cite{8767242}.

A arquitetura LoRaWAN, apesar de apresentar uma solução para o estabelecimento
de rede do sistema IoT, apresenta limitações como presença de nós
desconectados da rede em sistemas com poucos \textit{gateways} de cobertura e
falta de comunicação direta entre os nós \cite{8767242}. 
Tais limitações podem dificultar alguns tipos de sistemas IoT de longas 
distâncias, e dessa maneira ainda se fazem válidas as pesquisas e propostas de 
protocolos baseados em LoRa \cite{8767242}\cite{NAKAMURA2022257}.

\section{Objetivos}

O objetivo \textbf{geral} desse trabalho é propor, documentar e implementar um
protocolo de rede baseado em LoRa que forneça ao desenvolvedor de sistemas IoT
o suporte de rede para seus dispositivos inteligentes usando LoRa.

São objetivos \textbf{específicos} deste projeto:
\begin{itemize}
    \item Criação de documentação e diagramas sobre o protocolo proposto;
	\item Implementação do protocolo em forma de biblioteca;
	\item Execução de testes em dispositivos embarcados para validação do protocolo proposto;
	\item Avaliação dos resultados obtidos.
\end{itemize}

\section{Estrutura do Trabalho}

Neste capítulo são apresentados a motivação desse trabalho assim como o detalhamento dos
objetivos a serem alcançados, e a estruturação escolhida para o documento. Os próximos
capítulos abordados seguiram a seguinte estrutura:

\begin{itemize}
    \item \textbf{Capítulo 2}: Introdução de conceitos essenciais para o entendimento
    da proposta desse trabalho.
    
    \item \textbf{Capítulo 3}: Apresentação de alguns trabalhos relacionados
    e análise comparativa entre eles.
    
	\item \textbf{Capítulo 4}: Descrição do desenvolvimento da proposta, desde a
	idealização, por meio de metodólogas usadas em engenharia de software, até
	a implementação do projeto.
	
	\item \textbf{Capítulo 5}: Apresentação dos testes de software, descrição dos testes realizados sobre o protótipo em campo e análise dos resultados obtidos.
	
	\item \textbf{Capítulo 6}: Conclusão dos resultados obtidos durante o desenvolvimento
	da proposta e possíveis caminhos para o futuro deste trabalho.
	
\end{itemize}