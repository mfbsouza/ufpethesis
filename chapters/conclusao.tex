\chapter{Conclusão e Trabalhos Futuros}

Esse trabalho teve como objetivo propor, documentar e implementar um protocolo
de rede baseado na tecnologia LoRa para que desenvolvedores pudessem utilizá-lo
em seus sistemas embarcados indiferentemente dos hardwares e rádios utilizados.
Durante o desenvolvimento do projeto foram utilizadas bastantes metodologias da
literatura de engenharia de software como também foi buscado inspirações em tecnologias
já existentes em diversas áreas como sistemas embarcados, rede de computadores,
entre outras. Isso não só ajudou a garantir a qualidade do projeto como uma proposta
em si, mas também como um produto significativo para alguns cenários que foram
destacados nas etapas de idealização. Outro resultado destacável das metologias
utilizadas no desenvolvimento é qualidade da codificação do projeto principalmente
da perspectiva de código aberto, conquistando assim a possibilidade de evolução
continua da proposta não só pelo autor mas também por toda uma comunidade.
Os resultados apresentados nesse trabalho se mostraram no geral condizentes e
positivos com os objetivos definidos inicialmente, contudo é necessário a
realização de mais testes em campo e observando mais a fundo alguns detalhes
que não foram observados durante esse trabalho, como o tempo gasto para
diferentes tipos de transmissão, consumo de energia entre outros. O objetivo
disso é fortalecer mais ainda que todos os aspectos teóricos documentados
durante o desenvolvimento estejam presentes e funcionais na implementação.

\section{Análise comparativa com este trabalho}

Na tabela \ref{table:rel-comp-atual} a seguir, segue a comparação da tabela \ref{table:rel-comp} feita no capítulo de trabalhos relacionados agora atualizada com as características do projeto desenvolvido.

\begin{table}[H]
    \begin{center}
    \caption{Comparativo entre trabalhos relacionados e trabalho atual}
    \label{table:rel-comp-atual}
    \begin{tabular}{|l|l|l|l|}
    \hline
     & LoRaCTP & LoRaMesher & SNPES \\
    \hline
    Topologia de Rede & Nenhuma & Mesh & Estrela \\
    \hline
    Transmissão Confiável & Sim & Sim & Sim \\
    \hline
    Tratamento de Colisão & ALOHA & CSMA/CA & ALOHA \\
    \hline
    Empacotamento & Sim & Não & Sim \\
    \hline
    Integridade dos Dados & Checksum de 3 bytes & Nenhum & Nenhum \\
    \hline
    Independente do Rádio & Não & Não & Sim \\
    \hline
    Biblioteca para uso & Não & Sim & Sim \\
    \hline
    Código Aberto & Sim & Sim & Sim \\
    \hline
    \end{tabular}
    \end{center}
\end{table}

\section{Futuro do projeto}

A partir daqui os maiores objetivos desse projeto se tornam agora em trabalhar
para que cada vez mais ele pareça uma solução \textit{of the shelf} para o
desenvolvedor IoT. Dentre esses objetivos vale destacar as possíveis rotas futuras
para esse projeto:

\begin{itemize}
    \item \textbf{Testes em diferentes rádios:} Os rádios LoRaMESH da Radioenge utilizados para o desenvolvimento desse projeto podem ser considerados relativamente simples quando comparados a outros
    mais usados profissionalmente na área, os rádios usados possuem antenas simples e de baixo ganho
    do sinal, dificultando a visualização do desempenho real do projeto em cenários de longas
    distâncias. É objetivo do trabalho que se obtenha um bom desempenho para os mais
    diferentes rádios que o desenvolvedor possa querer utilizar, assim se faz importante
    a validação para esse distintos componentes de hardware.
    \item \textbf{Tratamento de colisão e Packet Loss:} Um tópico importante é o tratamento de colisões de pacotes, algo que é possível de acontecer na tecnologia LoRa e que pode causar um grande aumento de Packet Loss na rede entre outros problemas. foi implementando um tratamento de colisão relativamente simples baseado no protocolo ALOHA porém vale o estudo de outros protocolos como
    o CSMA/CD e CSMA/CA como inspirações para testar se eles podem ajudar mais ainda na diminuição
    de Packet Loss para o projeto desenvolvido.
    \item \textbf{Validação da integridade dos pacotes e Checksum:} Outro tópico importante na
    pauta de protocolos de rede é garantir a integridade dos pacotes, a tecnologia LoRa já
    possui embutida na sua modulação um CRC (Cyclic Redundancy Check) e por isso esse tópico
    não foi explorado no desenvolvimento do trabalho, porém para o futuro se faz valido estudar
    quais benefícios, além de tornar o protocolo na aparência mais robusto, e quais os custos de se implementar uma validação de integridade embutida no projeto.
    \item \textbf{Implementação em silício e Gateways IoT:} No atual estado o protocolo é totalmente
    desenvolvido em software e o hardware do Gateway compartilha das mesmas características dos Nós (um microcontrolador IoT). Um ponto interessante de estudo seria quais oportunidades possam surgir
    ao implementar o protocolo direto no silício, como num FPGA por exemplo, atuando como um coprocessador
    dentro de um Gateway IoT com hardware bem mais complexo que possa assim gerenciar não só uma rede,
    como feito no projeto mas múltiplas, pois durante o desenvolvimento do projeto é notável que o
    maior peso de processamento e de dados se encontra no Gateway.
    \item \textbf{Documentação de uso:} No estado atual o trabalho não possui nenhuma documentação 
    de uso para o desenvolvedor IoT de como ele pode adicionar o protocolo e como ele pode
    utilizado em seu projeto. Assim é importante desenvolver exemplos de como começar
    com o projeto, quais são as dependências, e códigos de exemplo para diversos
    casos de uso. Isso ajuda a remover a barreira inicial da adoção do projeto para
    os desenvolvedores.
    \item \textbf{Contribuição:} Investir em diminuir as barreiras iniciais que possam
    existir para que pessoas possam contribuir de diversas forma com o projeto, não só
    com codificação mas também com ideias. Isso trás um ponto extramente positivo que é
    garantir a vitalidade do projeto,isso significa que ele estará sempre evoluindo de
    alguma forma.
    \item \textbf{Validação continua:} Garantir que o projeto em seus diferentes
    estágios esteja sempre de acordo com uma \textit{pipeline} de testes. Isso garante
    a qualidade de produto para todos.
\end{itemize}