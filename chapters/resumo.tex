\resumo
Com o crescimento de aplicações de Big Data e Aprendizagem de Máquina em
áreas como Smart Cities, Smart Farming e Healthcare, a Internet das Coisas
tem o papel fundamental de viabilizadora desses projetos, o que intensifica a
indústria IoT a se manter em constante pesquisa de tecnologias como a LPWAN,
que permite o espalhamento de dispositivos inteligentes em
grandes áreas urbanas e rurais. Dentre elas a LoRa vem ganhando
destaque por suas características atrativas para um sistema IoT de longas distâncias. 
Entretanto a tecnologia LoRa não fornece ao desenvolvedor o suporte de uma rede dos 
dispositivos do seu sistema. Dessa forma, esse trabalho propõem
um protocolo baseado em LoRa para fornecer suporte de rede ao desenvolvedor de
Sistemas IoT.
\begin{keywords}
Sistemas Embarcados, Internet of Things, Low Power Wide Area Network, LoRa, Protocolo de Rede
\end{keywords}